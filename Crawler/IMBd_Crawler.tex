% Options for packages loaded elsewhere
\PassOptionsToPackage{unicode}{hyperref}
\PassOptionsToPackage{hyphens}{url}
\PassOptionsToPackage{dvipsnames,svgnames,x11names}{xcolor}
%
\documentclass[
  letterpaper,
  DIV=11,
  numbers=noendperiod]{scrartcl}

\usepackage{amsmath,amssymb}
\usepackage{lmodern}
\usepackage{iftex}
\ifPDFTeX
  \usepackage[T1]{fontenc}
  \usepackage[utf8]{inputenc}
  \usepackage{textcomp} % provide euro and other symbols
\else % if luatex or xetex
  \usepackage{unicode-math}
  \defaultfontfeatures{Scale=MatchLowercase}
  \defaultfontfeatures[\rmfamily]{Ligatures=TeX,Scale=1}
\fi
% Use upquote if available, for straight quotes in verbatim environments
\IfFileExists{upquote.sty}{\usepackage{upquote}}{}
\IfFileExists{microtype.sty}{% use microtype if available
  \usepackage[]{microtype}
  \UseMicrotypeSet[protrusion]{basicmath} % disable protrusion for tt fonts
}{}
\makeatletter
\@ifundefined{KOMAClassName}{% if non-KOMA class
  \IfFileExists{parskip.sty}{%
    \usepackage{parskip}
  }{% else
    \setlength{\parindent}{0pt}
    \setlength{\parskip}{6pt plus 2pt minus 1pt}}
}{% if KOMA class
  \KOMAoptions{parskip=half}}
\makeatother
\usepackage{xcolor}
\setlength{\emergencystretch}{3em} % prevent overfull lines
\setcounter{secnumdepth}{5}
% Make \paragraph and \subparagraph free-standing
\ifx\paragraph\undefined\else
  \let\oldparagraph\paragraph
  \renewcommand{\paragraph}[1]{\oldparagraph{#1}\mbox{}}
\fi
\ifx\subparagraph\undefined\else
  \let\oldsubparagraph\subparagraph
  \renewcommand{\subparagraph}[1]{\oldsubparagraph{#1}\mbox{}}
\fi

\usepackage{color}
\usepackage{fancyvrb}
\newcommand{\VerbBar}{|}
\newcommand{\VERB}{\Verb[commandchars=\\\{\}]}
\DefineVerbatimEnvironment{Highlighting}{Verbatim}{commandchars=\\\{\}}
% Add ',fontsize=\small' for more characters per line
\usepackage{framed}
\definecolor{shadecolor}{RGB}{241,243,245}
\newenvironment{Shaded}{\begin{snugshade}}{\end{snugshade}}
\newcommand{\AlertTok}[1]{\textcolor[rgb]{0.68,0.00,0.00}{#1}}
\newcommand{\AnnotationTok}[1]{\textcolor[rgb]{0.37,0.37,0.37}{#1}}
\newcommand{\AttributeTok}[1]{\textcolor[rgb]{0.40,0.45,0.13}{#1}}
\newcommand{\BaseNTok}[1]{\textcolor[rgb]{0.68,0.00,0.00}{#1}}
\newcommand{\BuiltInTok}[1]{\textcolor[rgb]{0.00,0.23,0.31}{#1}}
\newcommand{\CharTok}[1]{\textcolor[rgb]{0.13,0.47,0.30}{#1}}
\newcommand{\CommentTok}[1]{\textcolor[rgb]{0.37,0.37,0.37}{#1}}
\newcommand{\CommentVarTok}[1]{\textcolor[rgb]{0.37,0.37,0.37}{\textit{#1}}}
\newcommand{\ConstantTok}[1]{\textcolor[rgb]{0.56,0.35,0.01}{#1}}
\newcommand{\ControlFlowTok}[1]{\textcolor[rgb]{0.00,0.23,0.31}{#1}}
\newcommand{\DataTypeTok}[1]{\textcolor[rgb]{0.68,0.00,0.00}{#1}}
\newcommand{\DecValTok}[1]{\textcolor[rgb]{0.68,0.00,0.00}{#1}}
\newcommand{\DocumentationTok}[1]{\textcolor[rgb]{0.37,0.37,0.37}{\textit{#1}}}
\newcommand{\ErrorTok}[1]{\textcolor[rgb]{0.68,0.00,0.00}{#1}}
\newcommand{\ExtensionTok}[1]{\textcolor[rgb]{0.00,0.23,0.31}{#1}}
\newcommand{\FloatTok}[1]{\textcolor[rgb]{0.68,0.00,0.00}{#1}}
\newcommand{\FunctionTok}[1]{\textcolor[rgb]{0.28,0.35,0.67}{#1}}
\newcommand{\ImportTok}[1]{\textcolor[rgb]{0.00,0.46,0.62}{#1}}
\newcommand{\InformationTok}[1]{\textcolor[rgb]{0.37,0.37,0.37}{#1}}
\newcommand{\KeywordTok}[1]{\textcolor[rgb]{0.00,0.23,0.31}{#1}}
\newcommand{\NormalTok}[1]{\textcolor[rgb]{0.00,0.23,0.31}{#1}}
\newcommand{\OperatorTok}[1]{\textcolor[rgb]{0.37,0.37,0.37}{#1}}
\newcommand{\OtherTok}[1]{\textcolor[rgb]{0.00,0.23,0.31}{#1}}
\newcommand{\PreprocessorTok}[1]{\textcolor[rgb]{0.68,0.00,0.00}{#1}}
\newcommand{\RegionMarkerTok}[1]{\textcolor[rgb]{0.00,0.23,0.31}{#1}}
\newcommand{\SpecialCharTok}[1]{\textcolor[rgb]{0.37,0.37,0.37}{#1}}
\newcommand{\SpecialStringTok}[1]{\textcolor[rgb]{0.13,0.47,0.30}{#1}}
\newcommand{\StringTok}[1]{\textcolor[rgb]{0.13,0.47,0.30}{#1}}
\newcommand{\VariableTok}[1]{\textcolor[rgb]{0.07,0.07,0.07}{#1}}
\newcommand{\VerbatimStringTok}[1]{\textcolor[rgb]{0.13,0.47,0.30}{#1}}
\newcommand{\WarningTok}[1]{\textcolor[rgb]{0.37,0.37,0.37}{\textit{#1}}}

\providecommand{\tightlist}{%
  \setlength{\itemsep}{0pt}\setlength{\parskip}{0pt}}\usepackage{longtable,booktabs,array}
\usepackage{calc} % for calculating minipage widths
% Correct order of tables after \paragraph or \subparagraph
\usepackage{etoolbox}
\makeatletter
\patchcmd\longtable{\par}{\if@noskipsec\mbox{}\fi\par}{}{}
\makeatother
% Allow footnotes in longtable head/foot
\IfFileExists{footnotehyper.sty}{\usepackage{footnotehyper}}{\usepackage{footnote}}
\makesavenoteenv{longtable}
\usepackage{graphicx}
\makeatletter
\def\maxwidth{\ifdim\Gin@nat@width>\linewidth\linewidth\else\Gin@nat@width\fi}
\def\maxheight{\ifdim\Gin@nat@height>\textheight\textheight\else\Gin@nat@height\fi}
\makeatother
% Scale images if necessary, so that they will not overflow the page
% margins by default, and it is still possible to overwrite the defaults
% using explicit options in \includegraphics[width, height, ...]{}
\setkeys{Gin}{width=\maxwidth,height=\maxheight,keepaspectratio}
% Set default figure placement to htbp
\makeatletter
\def\fps@figure{htbp}
\makeatother

\KOMAoption{captions}{tableheading}
\makeatletter
\makeatother
\makeatletter
\makeatother
\makeatletter
\@ifpackageloaded{caption}{}{\usepackage{caption}}
\AtBeginDocument{%
\ifdefined\contentsname
  \renewcommand*\contentsname{Table of contents}
\else
  \newcommand\contentsname{Table of contents}
\fi
\ifdefined\listfigurename
  \renewcommand*\listfigurename{List of Figures}
\else
  \newcommand\listfigurename{List of Figures}
\fi
\ifdefined\listtablename
  \renewcommand*\listtablename{List of Tables}
\else
  \newcommand\listtablename{List of Tables}
\fi
\ifdefined\figurename
  \renewcommand*\figurename{Figure}
\else
  \newcommand\figurename{Figure}
\fi
\ifdefined\tablename
  \renewcommand*\tablename{Table}
\else
  \newcommand\tablename{Table}
\fi
}
\@ifpackageloaded{float}{}{\usepackage{float}}
\floatstyle{ruled}
\@ifundefined{c@chapter}{\newfloat{codelisting}{h}{lop}}{\newfloat{codelisting}{h}{lop}[chapter]}
\floatname{codelisting}{Listing}
\newcommand*\listoflistings{\listof{codelisting}{List of Listings}}
\makeatother
\makeatletter
\@ifpackageloaded{caption}{}{\usepackage{caption}}
\@ifpackageloaded{subcaption}{}{\usepackage{subcaption}}
\makeatother
\makeatletter
\@ifpackageloaded{tcolorbox}{}{\usepackage[many]{tcolorbox}}
\makeatother
\makeatletter
\@ifundefined{shadecolor}{\definecolor{shadecolor}{rgb}{.97, .97, .97}}
\makeatother
\makeatletter
\makeatother
\ifLuaTeX
  \usepackage{selnolig}  % disable illegal ligatures
\fi
\IfFileExists{bookmark.sty}{\usepackage{bookmark}}{\usepackage{hyperref}}
\IfFileExists{xurl.sty}{\usepackage{xurl}}{} % add URL line breaks if available
\urlstyle{same} % disable monospaced font for URLs
\hypersetup{
  pdftitle={IMBd Crawler},
  colorlinks=true,
  linkcolor={blue},
  filecolor={Maroon},
  citecolor={Blue},
  urlcolor={Blue},
  pdfcreator={LaTeX via pandoc}}

\title{IMBd Crawler}
\author{}
\date{}

\begin{document}
\maketitle
\ifdefined\Shaded\renewenvironment{Shaded}{\begin{tcolorbox}[breakable, boxrule=0pt, borderline west={3pt}{0pt}{shadecolor}, sharp corners, enhanced, frame hidden, interior hidden]}{\end{tcolorbox}}\fi

\renewcommand*\contentsname{Table of contents}
{
\hypersetup{linkcolor=}
\setcounter{tocdepth}{3}
\tableofcontents
}
Hello everyone, making \textbf{Quarto} or \textbf{Jupyter} projects will
help me to show the process of creating \textbf{R} or \textbf{Python}
projects with the advantage to make them easy to follow.

The following packages will be used for this project, so make sure to
install them.

\begin{Shaded}
\begin{Highlighting}[]
\FunctionTok{library}\NormalTok{(rvest)}
\FunctionTok{library}\NormalTok{(lubridate)}
\FunctionTok{library}\NormalTok{(tidyverse)}
\end{Highlighting}
\end{Shaded}

\hypertarget{crawling-through-imdb}{%
\subsection{Crawling through IMDb}\label{crawling-through-imdb}}

The first part of crawling or scraping through IMDb is to know how IMDb
gets the information to display it, you can get a general idea from the
\emph{URL.}

After that, we can break the process of gathering all the information of
all the seasons into getting the information of one season at the time
and then loop through the next season until we reach the last one.

Retrieving the information of a single season will be one function, and
loop through all will be another function that uses the first one.

\begin{quote}
Having loops making everything can be considered as a bad practice.

Functions are generally used to break a big process into small pieces.
\end{quote}

\hypertarget{the-url}{%
\paragraph{The URL}\label{the-url}}

The format to crawl though any TV series in IMDb is:

\begin{quote}
\textbf{https://www.imdb.com/title/} + \textless{}\textbf{id of the
series\textgreater{}} + \textbf{episodes?season= + \textless season
number\textgreater{}}
\end{quote}

\hypertarget{scraping-single-season}{%
\subsubsection{Scraping Single Season}\label{scraping-single-season}}

As described above, the first thing to do is crawl and retrieve the
information only of one season.

This is the function of \emph{season\_data}. It gets the html contents
for a given series and season.

After that, it parses as text the information about each episode and
return a tibble containing that information.

\begin{quote}
Other way to read the statements above is: Get the information about the
episodes and return a table with that information.
\end{quote}

\begin{Shaded}
\begin{Highlighting}[]
\CommentTok{\# The function season\_data retrieves the html for a given series and season.}
\CommentTok{\# After that, it parse as text the information about each episode and return a tibble containing that information.}
\NormalTok{season\_data }\OtherTok{\textless{}{-}} \ControlFlowTok{function}\NormalTok{(season\_url, season\_number)\{}
\NormalTok{  current\_url }\OtherTok{\textless{}{-}}  \FunctionTok{str\_c}\NormalTok{(season\_url, season\_number)}
  
\NormalTok{  season\_serie }\OtherTok{\textless{}{-}} \FunctionTok{read\_html}\NormalTok{(current\_url) }\SpecialCharTok{|\textgreater{}} 
    \FunctionTok{html\_elements}\NormalTok{(}\StringTok{".list\_item"}\NormalTok{)}
  
\NormalTok{  seasons\_episodes }\OtherTok{\textless{}{-}} \FunctionTok{tibble}\NormalTok{(}\AttributeTok{Season =}\NormalTok{ season\_number}
\NormalTok{                             , }\AttributeTok{episode\_air\_date =}\NormalTok{ season\_serie }\SpecialCharTok{|\textgreater{}} \FunctionTok{html\_element}\NormalTok{(}\StringTok{".airdate"}\NormalTok{) }\SpecialCharTok{|\textgreater{}} \FunctionTok{html\_text2}\NormalTok{()}
\NormalTok{                             , }\AttributeTok{episode\_name =}\NormalTok{ season\_serie }\SpecialCharTok{|\textgreater{}} \FunctionTok{html\_element}\NormalTok{(}\StringTok{"strong"}\NormalTok{) }\SpecialCharTok{|\textgreater{}} \FunctionTok{html\_text2}\NormalTok{()}
\NormalTok{                             , }\AttributeTok{episode\_rate =}\NormalTok{ season\_serie }\SpecialCharTok{|\textgreater{}} \FunctionTok{html\_element}\NormalTok{(}\StringTok{".ipl{-}rating{-}star\_\_rating"}\NormalTok{) }\SpecialCharTok{|\textgreater{}} \FunctionTok{html\_text2}\NormalTok{()}
\NormalTok{                             , }\AttributeTok{episode\_votes =}\NormalTok{ season\_serie }\SpecialCharTok{|\textgreater{}} \FunctionTok{html\_element}\NormalTok{(}\StringTok{".ipl{-}rating{-}star\_\_total{-}votes"}\NormalTok{) }\SpecialCharTok{|\textgreater{}} \FunctionTok{html\_text2}\NormalTok{()}
\NormalTok{                             , }\AttributeTok{episode\_description =}\NormalTok{ season\_serie }\SpecialCharTok{|\textgreater{}} \FunctionTok{html\_element}\NormalTok{(}\StringTok{".item\_description"}\NormalTok{) }\SpecialCharTok{|\textgreater{}} \FunctionTok{html\_text2}\NormalTok{()}
\NormalTok{  )}
  
  \FunctionTok{return}\NormalTok{(seasons\_episodes)}
\NormalTok{\}}
\end{Highlighting}
\end{Shaded}

\hypertarget{scraping-all-seasons}{%
\subsubsection{Scraping All Seasons}\label{scraping-all-seasons}}

Now, we have the general idea of \emph{how to get the information about
only a single season,} but we need to loop through several seasons. That
is what \emph{all\_seasons} do, it goes through each season, calls
\emph{season\_data} until we reach the season we want.

\begin{quote}
Other way to read the statements above is: Go season through season and
add it to the table containing all the seasons information.
\end{quote}

\begin{Shaded}
\begin{Highlighting}[]
\NormalTok{all\_seasons }\OtherTok{\textless{}{-}} \ControlFlowTok{function}\NormalTok{(url, num\_seasons)\{}
\NormalTok{  all\_seasons }\OtherTok{\textless{}{-}} \FunctionTok{tibble}\NormalTok{()}

  \ControlFlowTok{for}\NormalTok{(season }\ControlFlowTok{in} \DecValTok{1}\SpecialCharTok{:}\NormalTok{num\_seasons)\{}
\NormalTok{    all\_seasons }\OtherTok{\textless{}{-}} \FunctionTok{bind\_rows}\NormalTok{(all\_seasons, }\FunctionTok{season\_data}\NormalTok{(url, season))}
\NormalTok{  \}}

  \FunctionTok{return}\NormalTok{(all\_seasons)}
\NormalTok{\}}
\end{Highlighting}
\end{Shaded}

\hypertarget{examples}{%
\subsection{Examples}\label{examples}}

Now you can go to IMDb and search for any series, I will show two
examples of to know series.

\hypertarget{the-joy-of-painting-seasons-1-3}{%
\paragraph{The Joy of Painting (Seasons
1-3)}\label{the-joy-of-painting-seasons-1-3}}

The Joy of Painting, can we say anything more that beautiful oil
paintings on canvas by \textbf{Bob Ross?}

\begin{Shaded}
\begin{Highlighting}[]
\NormalTok{Joy\_Painting }\OtherTok{\textless{}{-}} \FunctionTok{all\_seasons}\NormalTok{(}\StringTok{"https://www.imdb.com/title/tt0383795/episodes?season="}\NormalTok{, }\DecValTok{3}\NormalTok{)}

\NormalTok{Joy\_Painting}
\end{Highlighting}
\end{Shaded}

\begin{verbatim}
# A tibble: 39 x 6
   Season episode_air_date episode_name        episode_rate episode_vo~1 episo~2
    <int> <chr>            <chr>               <chr>        <chr>        <chr>  
 1      1 11 Jan. 1983     A Walk in the Woods 9.1          (91)         "Bob R~
 2      1 11 Jan. 1983     Mt. McKinley        9.4          (74)         "Bob p~
 3      1 18 Jan. 1983     Ebony Sunset        9.2          (62)         "Bob u~
 4      1 25 Jan. 1983     Winter Mist         9.2          (56)         "Bob p~
 5      1 1 Feb. 1983      Quiet Stream        9.1          (55)         "Bob p~
 6      1 8 Feb. 1983      Winter Moon         9.4          (53)         "Anoth~
 7      1 15 Feb. 1983     Autumn Mountains    9.4          (49)         "An al~
 8      1 22 Feb. 1983     Peaceful Valley     9.4          (48)         "Bob p~
 9      1 1 Mar. 1983      Seascape            9.3          (51)         "Bob p~
10      1 8 Mar. 1983      Mountain Lake       9.4          (50)         "Bob p~
# ... with 29 more rows, and abbreviated variable names 1: episode_votes,
#   2: episode_description
\end{verbatim}

\hypertarget{formula-1-drive-to-surviveseasons-1-5}{%
\paragraph{Formula 1: Drive to Survive(Seasons
1-5)}\label{formula-1-drive-to-surviveseasons-1-5}}

\textbf{F1 documentary}, amazing work to know more about the drivers,
teams, etc. Lots of drama.

\begin{Shaded}
\begin{Highlighting}[]
\NormalTok{F1\_drive }\OtherTok{\textless{}{-}} \FunctionTok{all\_seasons}\NormalTok{(}\StringTok{"https://www.imdb.com/title/tt8289930/episodes/?season="}\NormalTok{, }\DecValTok{2}\NormalTok{)}

\NormalTok{F1\_drive}
\end{Highlighting}
\end{Shaded}

\begin{verbatim}
# A tibble: 20 x 6
   Season episode_air_date episode_name         episode_rate episode_v~1 episo~2
    <int> <chr>            <chr>                <chr>        <chr>       <chr>  
 1      1 8 Mar. 2019      All to Play For      7.8          (1,141)     Driver~
 2      1 8 Mar. 2019      The King of Spain    7.7          (998)       Team M~
 3      1 8 Mar. 2019      Redemption           8.3          (1,012)     At the~
 4      1 8 Mar. 2019      The Art of War       8.1          (940)       The tr~
 5      1 8 Mar. 2019      Trouble at the Top   7.6          (862)       A team~
 6      1 8 Mar. 2019      All or Nothing       7.8          (874)       When F~
 7      1 8 Mar. 2019      Keeping Your Head    7.6          (833)       Perhap~
 8      1 8 Mar. 2019      The Next Generation  8.1          (853)       Sauber~
 9      1 8 Mar. 2019      Stars and Stripes    7.6          (810)       The bi~
10      1 8 Mar. 2019      Crossing the Line    7.7          (814)       Driver~
11      2 28 Feb. 2020     Lights Out           7.6          (860)       The 20~
12      2 28 Feb. 2020     Boiling Point        7.9          (788)       As med~
13      2 28 Feb. 2020     Dogfight             7.7          (794)       Carlos~
14      2 28 Feb. 2020     Dark Days            8.2          (855)       Lewis ~
15      2 28 Feb. 2020     Great Expectations   7.9          (792)       Red Bu~
16      2 28 Feb. 2020     Raging Bulls         8.6          (939)       Alex A~
17      2 28 Feb. 2020     Seeing Red           7.7          (759)       Charle~
18      2 28 Feb. 2020     Musical Chairs       7.7          (742)       Niko H~
19      2 28 Feb. 2020     Blood, Sweat & Tears 7.5          (731)       Team W~
20      2 28 Feb. 2020     Checkered Flag       8.2          (785)       Pierre~
# ... with abbreviated variable names 1: episode_votes, 2: episode_description
\end{verbatim}

\hypertarget{data-cleansing}{%
\subsection{Data Cleansing}\label{data-cleansing}}

If you remember, we scrapped all the data as \emph{text} and we can not
work at all with this format. We need to clean and transform the data
into the correct shape and format.

\begin{Shaded}
\begin{Highlighting}[]
\NormalTok{clean\_seasons }\OtherTok{\textless{}{-}} \ControlFlowTok{function}\NormalTok{(seasons\_table)\{}
\NormalTok{  seasons\_table }\OtherTok{\textless{}{-}}\NormalTok{ seasons\_table }\SpecialCharTok{|\textgreater{}} 
    \FunctionTok{mutate}\NormalTok{(}\AttributeTok{episode\_air\_date =} \FunctionTok{dmy}\NormalTok{(episode\_air\_date),}
           \AttributeTok{episode\_rate =} \FunctionTok{parse\_number}\NormalTok{(episode\_rate),}
           \AttributeTok{episode\_votes =} \FunctionTok{parse\_number}\NormalTok{(episode\_votes)}
\NormalTok{           ) }\SpecialCharTok{|\textgreater{}} 
    \FunctionTok{group\_by}\NormalTok{(Season) }\SpecialCharTok{|\textgreater{}} 
    \FunctionTok{mutate}\NormalTok{(}\AttributeTok{episode =} \FunctionTok{row\_number}\NormalTok{()) }\SpecialCharTok{|\textgreater{}} 
    \FunctionTok{select}\NormalTok{(Season, episode, }\FunctionTok{everything}\NormalTok{())}
  
  \FunctionTok{return}\NormalTok{(seasons\_table)}
\NormalTok{\}}
\end{Highlighting}
\end{Shaded}

\hypertarget{cleansing-the-f1-data}{%
\subsubsection{Cleansing the F1 data}\label{cleansing-the-f1-data}}

\begin{Shaded}
\begin{Highlighting}[]
\NormalTok{F1\_drive }\OtherTok{\textless{}{-}} \FunctionTok{clean\_seasons}\NormalTok{(F1\_drive)}

\NormalTok{F1\_drive}
\end{Highlighting}
\end{Shaded}

\begin{verbatim}
# A tibble: 20 x 7
# Groups:   Season [2]
   Season episode episode_air_date episode_name         episod~1 episo~2 episo~3
    <int>   <int> <date>           <chr>                   <dbl>   <dbl> <chr>  
 1      1       1 2019-03-08       All to Play For           7.8    1141 Driver~
 2      1       2 2019-03-08       The King of Spain         7.7     998 Team M~
 3      1       3 2019-03-08       Redemption                8.3    1012 At the~
 4      1       4 2019-03-08       The Art of War            8.1     940 The tr~
 5      1       5 2019-03-08       Trouble at the Top        7.6     862 A team~
 6      1       6 2019-03-08       All or Nothing            7.8     874 When F~
 7      1       7 2019-03-08       Keeping Your Head         7.6     833 Perhap~
 8      1       8 2019-03-08       The Next Generation       8.1     853 Sauber~
 9      1       9 2019-03-08       Stars and Stripes         7.6     810 The bi~
10      1      10 2019-03-08       Crossing the Line         7.7     814 Driver~
11      2       1 2020-02-28       Lights Out                7.6     860 The 20~
12      2       2 2020-02-28       Boiling Point             7.9     788 As med~
13      2       3 2020-02-28       Dogfight                  7.7     794 Carlos~
14      2       4 2020-02-28       Dark Days                 8.2     855 Lewis ~
15      2       5 2020-02-28       Great Expectations        7.9     792 Red Bu~
16      2       6 2020-02-28       Raging Bulls              8.6     939 Alex A~
17      2       7 2020-02-28       Seeing Red                7.7     759 Charle~
18      2       8 2020-02-28       Musical Chairs            7.7     742 Niko H~
19      2       9 2020-02-28       Blood, Sweat & Tears      7.5     731 Team W~
20      2      10 2020-02-28       Checkered Flag            8.2     785 Pierre~
# ... with abbreviated variable names 1: episode_rate, 2: episode_votes,
#   3: episode_description
\end{verbatim}

\hypertarget{whats-next}{%
\subsection{What's Next?}\label{whats-next}}

Now, the information about all our seasons is clean and ready to be
upload to a database, csv file, Excel file or any other file extension
format.

Other things to improve is to allow the users type the name of the
series and return the id of the series, maybe with \textbf{RSelenium} or
similar packages.

Have fun!.



\end{document}
